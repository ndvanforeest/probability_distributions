\documentclass[assignments]{subfiles}

\begin{document}


\section{Assignment 5, TBD}

\subsection{Have you read well?}

\section{Bayes' Billiards}
\label{sec:orgf93559e}
Take \(n=100\) samples.

\subsection{One coin}
\label{sec:org1e9a031}
\begin{enumerate}
\item Take success probability \(p=1/2\).
\item Make matrix with \(n\) rows, \(n\) columns. Each row is an experiment of \(n\) throws of the coin.
\item Plot the histogram of the number of heads.
\end{enumerate}

\subsection{Three coins}
\label{sec:orga657159}
\begin{enumerate}
\item Take three coins with success probabilities \(p=1/4, 1/2, 3/4\).
\item Make a simulation for each coin.
\item If we select a coin with probability \(1/3\), the total histogram is the 1/3 times the sum of the histograms of each of the coins. That is, \(\P{X=k} = \P{X=k|C = i}\P{C=i}\), where \(C\) is one of chosen coins;  here we take \(\P{C=i} = 1/3\).
\end{enumerate}

\subsection{five coins}
\label{sec:orga1a3b28}
\begin{enumerate}
\item Select with uniform probability one out of five coins with success probabilities \(p=i/5\),  \(i=1,\ldots, 5\).
\item Make a simulation for each coin.
\item Make the histogram \(\P{X=k} = \P{X=k|C = i}\P{C=i}\), where \(\P{C=i} = 1/5\).
\end{enumerate}


\subsection{Coding skills}
\label{sec:coding-skills-1}

\paragraph{The mystery box}

We use  simulation to solve  BH.9.7.
Read it now, i.e., before reading the text below, then read the code below.


In the questions we ask you to explain what the code does.
There are lots of print statements that have been commented out, but we left them in for you to include while experimenting with the code to see how the code works.
(I often use print statements of intermediate results when writing a program, just to see whether I am still on track.
Once I checked, I remove them, because they clutter the looks of the code.)

\begin{minted}[]{python}

\end{minted}


\begin{minted}[]{python}
import numpy as np
from scipy.stats import uniform
import matplotlib.pyplot as plt

np.random.seed(3)


N = 1000
a, b = 0, 1000_000
V = uniform(a, b).rvs(N)

x_range = np.linspace(b / 5, b / 2, num=50)
y = np.zeros_like(x_range)

for i, b in enumerate(x_range):
    payoff = (V - b) * (b >= V / 4)
    y[i] = payoff.mean()


plt.plot(x_range, y)
plt.show()
\end{minted}

As an aside, this code is amazingly short, isn't it?

\begin{exercise}
Use the scipy documentation to explain why $V\sim\Unif{[0,10^{6}]}$.
\begin{solution}
\end{solution}
\end{exercise}



\begin{exercise}
What are the smallest and the largest value of \verb|x_range|.
\begin{solution}
\end{solution}
\end{exercise}


\begin{exercise}
Change the acceptance threshold from to $V/4$ to $V/5$ (or $V/6$, what ever you like), and make a graph of the pay offs. Include the graph in your report.
\begin{solution}
\end{solution}
\end{exercise}

\begin{exercise}
or the payoff to e.g $\sqrt{V-b}$, or some other weird function like the sine, what is the influence on the mean and std of the payoff?
\begin{solution}
\end{solution}
\end{exercise}



\begin{exercise}
For N small, e.g. N=10, you can get quite strange values. Why is that?
\begin{solution}
\end{solution}
\end{exercise}


\begin{exercise}
Suppose after seeing the graph of the payoffs, and this graph would only increase, or decrease, how would you change \verb|x_range|?
\begin{solution}
\end{solution}
\end{exercise}


\end{document}
