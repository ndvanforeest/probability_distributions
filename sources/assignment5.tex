\documentclass[assignments]{subfiles}

\begin{document}


\section{Assignment 5, TBD}

\subsection{Have you read well?}

\section{Bayes' Billiards}
\label{sec:orgf93559e}
Take \(n=100\) samples.

\subsection{One coin}
\label{sec:org1e9a031}
\begin{enumerate}
\item Take success probability \(p=1/2\).
\item Make matrix with \(n\) rows, \(n\) columns. Each row is an experiment of \(n\) throws of the coin.
\item Plot the histogram of the number of heads.
\end{enumerate}

\subsection{Three coins}
\label{sec:orga657159}
\begin{enumerate}
\item Take three coins with success probabilities \(p=1/4, 1/2, 3/4\).
\item Make a simulation for each coin.
\item If we select a coin with probability \(1/3\), the total histogram is the 1/3 times the sum of the histograms of each of the coins. That is, \(\P{X=k} = \P{X=k|C = i}\P{C=i}\), where \(C\) is one of chosen coins;  here we take \(\P{C=i} = 1/3\).
\end{enumerate}

\subsection{five coins}
\label{sec:orga1a3b28}
\begin{enumerate}
\item Select with uniform probability one out of five coins with success probabilities \(p=i/5\),  \(i=1,\ldots, 5\).
\item Make a simulation for each coin.
\item Make the histogram \(\P{X=k} = \P{X=k|C = i}\P{C=i}\), where \(\P{C=i} = 1/5\).
\end{enumerate}


\subsection{Coding skills}
\label{sec:coding-skills-1}

\paragraph{The mystery box}

We use  simulation to solve  BH.9.7.
Read it now, i.e., before reading the text below, then read the code below.
Note how short  this code is;  amazing, isn't it?



\begin{minted}[]{python}
import numpy as np
from scipy.stats import uniform
import matplotlib.pyplot as plt

np.random.seed(3)


N = 1000
a, b = 0, 1000_000
V = uniform(a, b).rvs(N)

x_range = np.linspace(b / 5, b / 2, num=50)
y = np.zeros_like(x_range)

for i, b in enumerate(x_range):
    payoff = (V - b) * (b >= V / 4)
    y[i] = payoff.mean()


plt.plot(x_range, y)
plt.show()
\end{minted}


\begin{minted}[]{R}

\end{minted}


\begin{exercise}
Use the scipy documentation to explain why $V\sim\Unif{[0,10^{6}]}$.
\begin{solution}
\end{solution}
\end{exercise}



\begin{exercise}
What are the smallest and the largest value of \verb|x_range|?
\begin{solution}
\end{solution}
\end{exercise}

\begin{exercise}
Run the code above and make a graph. Include the graph in your report, and explain what you see in the graph. For instance, is there a maximum? If so, can you explain where the maximum occurs? Can you explain how the maximum should be?
\begin{solution}
\end{solution}
\end{exercise}


\begin{exercise}
Suppose after seeing the graph of the payoffs, and this graph would only increase, or decrease, how would you change \verb|x_range|? Do you expect to see a maximum?
\begin{solution}
\end{solution}
\end{exercise}




\begin{exercise}
For N small, e.g. N=10, you can get quite strange values. Why is that?
\begin{solution}
\end{solution}
\end{exercise}


\begin{exercise}
Change the acceptance threshold from to $V/4$ to $V/5$ (or $V/6$, or some other value you like), and make a graph of the payoffs.
Include the graph in your report.
\begin{solution}
\end{solution}
\end{exercise}

\begin{exercise}
Change the payoff function to e.g $\sqrt{V-b}$, or some weird function that you like particularly such as $\sin |V-b|$ (any non-trivial function goes).
Make a graph of the  mean and std of the payoff. Can you explain your graph?
\begin{solution}
\end{solution}
\end{exercise}

\paragraph{Kelly makes bets}



\end{document}
