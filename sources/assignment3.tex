\documentclass[assignments]{subfiles}

\begin{document}


\section{Assignment 3, TBD}
\label{sec:assignment-3}

Topics of chapter 8.

\subsection{Have you read well?}
\label{sec:have-you-read-1}

\begin{exercise}
Explain in your own words:
\begin{enumerate}
\item What is a prior?
\item What is a conjugate prior?
\end{enumerate}
\end{exercise}

\begin{exercise}
Look up on the web: what is the conjugate prior of the multinomial distribution?
\end{exercise}



\begin{exercise}
Henri Poincaré was a French mathematician who taught at the Sorbonne around 1900.
The following anecdote about him is probably fabricated, but it makes an interesting probability problem.
Supposedly Poincaré suspected that his local bakery was selling loaves of bread that were lighter than the advertised weight of 1 kg, so every day for a year he bought a loaf of bread, brought it home and weighed it.
At the end of the year, he plotted the distribution of his measurements and showed that it fit a normal distribution with mean 950 g and standard deviation 50 g.
He brought this evidence to the bread police, who gave the baker a warning.
For the next year, Poincaré continued the practice of weighing his bread every day.
At the end of the year, he found that the average weight was 1000 g, just as it should be, but again he complained to the bread police, and this time they fined the baker.


Why?

\begin{solution}
Because the shape of the distribution was asymmetric. Unlike the normal distribution, it was skewed to the right, which is consistent with the hypothesis that the baker was still making 950 g loaves, but deliberately giving Poincaré the heavier ones.
Exercise 5.6 Write a program that simulates a baker who chooses n loaves from a distribution with mean 950 g and standard deviation 50 g, and gives the heaviest one to Poincaré. What value of n yields a distribution with mean 1000 g? What is the standard deviation?
\end{solution}

\end{exercise}


\subsection{About exam level}
\label{sec:chapter-8}

\begin{exercise}
Let $X, Y$ iid $\sim \Unif{[0,1]}$.
\begin{enumerate}
\item What is the joint CDF of $X+Y, XY$?
\item What is the joint PDF of $X+Y, XY$?
\item Compute $\P{X+Y\leq 1, XY\leq 2/9}$.
\end{enumerate}
\end{exercise}

\begin{exercise}
Let $X, Y$ be continuous r.v.s with CDF $F_{X,Y}(x,y) = (x-1)^{2}(y-2)/8$ for $x \in (1, 3)$.
\begin{enumerate}[a.]
\item Explain that $y\in (2,4)$ for  $F$ to be a proper CDF.
\item What is $F(3,7)$?
\item Determine the PDF.
\item Compute $\P{2< X < 3}$
\item Compute $\P{2< X < 3, 2<Y<4}$.
\item Compute $\P{Y< 2X}$.
\item Compute $\P{Y\leq  2X}$.
\item Compute $\P{Y< 2X, Y+2X > 6}$.
\end{enumerate}
\begin{solution}
a.
\begin{align}
\label{eq:11}
F \geq 0 &\implies 2<y\\
F \leq 1 &\implies F(3, y)\leq 1 \implies F(3,4)=1
\end{align}

b. $F(3,7) = 1$.

c. $f(x,y) = \partial_{x} \partial_{y} F(x,y) = (x-1)/4$ for $x\in(1,3), y\in(2,4)$ and $0$ elsewhere.

d.
\begin{align}
  \label{}
\P{2 < X < 3}
&= F_{X}(3) - F_{X}(2) \\
&= F_{X,Y}(3, 4) - F_{X,Y}(2,4) = 1 - 1\cdot 2/8 = 3/4.
\end{align}

e.
Make a drawing of the rectangle $[2,3]\times[2,4]$. Then check what parts of this are covered by $F_{X,Y}$.
\begin{align}
  \label{}
\P{2 < X < 3, 2 < Y < 4}
&= F_{X,Y}(3, 4) - F_{X,Y}(2,4)  - F_{X,Y}(3, 2) + F_{X,Y}(2,2).
\end{align}
The rest is just number plugging.

f.
Use the fundamental bridge and c.
\begin{align}
\P{Y< 2 X}
&= \E{\1{Y < 2 X}} \\
&= \iint \1{y < 2 x} f_{X,Y}(x,y) \d x \d y\\
&= \frac{1}{4}\iint \1{y < 2 x} \1{2< y <4} \1{1<x<3} (x-1) \d x \d y\\
&= \frac{1}{4}\int_{1}^{3} (x-1) \int \1{2<y < \min\{2 x,4\}}  \d y \d x \\
&= \frac{1}{4}\int_{1}^{3} (x-1) (\min\{2 x,4\}-2) \d x \\
&= \frac{1}{4}\int_{1}^{2} (x-1) (2 x-2) \d x
+ \frac{1}{4}\int_{2}^{3} (x-1) (4-2) \d x.
\end{align}
Finishing the computation must be easy for you now (and if not, practice real hard).

g. As $X, Y$ continuous, the answer is equal to that of f.

h. Similar to f. but a bit more involved.
\begin{align}
  \P{Y< 2 X, Y + 2X > 6}
  &= \E{\1{Y < 2 X, Y > 6 - 2X}} \\
  &= \iint \1{y < 2 x, y> 6-2x} f_{X,Y}(x,y) \d x \d y\\
  &= \frac{1}{4}\iint \1{y < 2 x, y>6-2x} \1{2< y <4} \1{1<x<3} (x-1) \d x \d y\\
  &= \frac{1}{4}\int_{1}^{3}(x-1) \int \1{\max\{2, 6 -2x\} < y < \min\{2 x, 4\}} \d y \d x\\
  &= \frac{1}{4}\int_{1}^{3}(x-1) [\min\{2x, 4\} - \max\{2, 6 -2x\}]^{+} \d x,
    \intertext{where we need the $[\cdot]^{+}$  to ensure the positivity of $\min\{2x, 4\} - \max\{2, 6 -2x\}$. To see this, make a graph of  the function $\min\{2x, 4\} - \max\{2, 6 -2x\}$. Also, from this graph,}
  &= \frac{1}{4}\int_{3/2}^{2}(x-1) (2x - 6 + 2x)  \d x + \frac{1}{4}\int_{2}^{3}(x-1) (4-2) \d x.
\end{align}
The rest is for you.
\end{solution}
\end{exercise}




\subsection{Coding skills}
\label{sec:coding-skills-1}

\paragraph{Ping pong balls}



How many ping pong balls fit into an Airbus Beluga?
One way to answer this is as follows.
According to this \href{https://en.wikipedia.org/wiki/Airbus\_Beluga}{wikipage} the cargo volume $V$ of this airplane is $1500 \m^{3}$.
But this number is based on the physical dimensions that is available to store containers, tanks, and so on.
So, I estimate the volume as about twice that amount, i.e., $V = 2500 \m^{3}$.
The volume of a ping pong ball is $v = 4 \pi r^3/3  = \py{4*3.14*8/3} \cm^{3}$ with $r=2$ cm.
A plain division gives \py{2500/33.5} ping pong balls.
Note, I left out the $10^{6}$ conversion from meters to cm, and I do not take into the sphere packing factor.
Besides that, I hope you agree with me that providing an result with the precision as given here is plain ridiculous.
(But from reason incomprehensible to me, even professional econometricians like to report results with 10 digits or more, without questioning the precision.)


However, I know that the volumes of an air plane and a ping pong ball is an estimate, rather than a precise number as assumed above.
It seems to be better to approximate $V$ and $v$ as rvs.
Let's assume that
   \begin{align*}
V & \sim N(2500, 500^{2}), & v  & \sim N(33.5, \sigma=0.5^{2}),
\end{align*}
where the variances express my trust in my guess work.
What is now the mean of $N = V/v$ and its std?
In fact, finding the closed form expression for the distribution of $N$ is not entirely simple.
However, with simulation it's easy to get an estimate.

\begin{exercise}\label{ex:2}
 How does this exercise relate to BH.8.11 and BH.8.12? What is similar, what are crucial differences?
\end{exercise}

\begin{exercise}
Use the documentation of the \texttt{norm} (\texttt{rnorm})function of python (R) to explain why we set the scale as we do.
Relate this to location-scale discussion in BH.

\begin{solution}
\end{solution}
\end{exercise}

\begin{exercise}
Explain lines 8 and 11 of the python code or lines 5 and 8 of the R code. (If you are interested in both languages, also comment on the difference)
\begin{solution}
\end{solution}
\end{exercise}

\begin{exercise}
Use the code below to provide the estimates.
\begin{solution}
\end{solution}
\end{exercise}

\begin{exercise}
Contrary to BH.7.1.25 if you run the code below, you'll see that $\E N < \infty$, and, in fact, very near to the deterministic answer.
But isn't this strange?
We divide two normal random variables, just like BH.7.1.25, but there the expectation is infinite.
Comment on the difference.
\begin{solution}
\end{solution}
\end{exercise}


The numerical results suggest the interesting guess $\V N \approx \V V * \V v$, but is this true more generally?
In~\cref{sec:challenges-1} we study this problem in more detail.

\begin{minted}[]{python}
import numpy as np
from scipy.stats import norm

num = 500

np.random.seed(3)

V = norm(loc=2500, scale=500)
v = norm(loc=33.5, scale=0.5)

print(V.mean(), V.std()) # just a check

N = V.rvs(num) / v.rvs(num)
print(N.mean(), N.std())

print(2500/33.5)
print(np.sqrt(500*0.5))
\end{minted}

\begin{minted}[]{R}
num <- 500

set.seed(3)

V = rnorm(num, 2500, 500)
v = rnorm(num, 33.5, 0.5)

N = V / v
paste(mean(N), sd(N))

2500/33.5
sqrt(500*0.5)
\end{minted}



\paragraph{Sums of RVS}

We start from BH.8.27 (which you have to read now).  We are interested in the difference between the distribution of $X+Y+Z$ and the normal distribution. But why the normal distribution? As it turns out, the central limit law, see BH.10, states that the distribution of sums of r.v.s converge to the normal distribution (in a specific sense)

Here some code to simulate.

\begin{minted}[]{python}
import numpy as np
from scipy.stats import norm

import matplotlib.pylab as plt
import seaborn as sns

sns.set()

np.random.seed(3)

k = 3
Zexact = norm(loc=k / 2, scale=np.sqrt(k / 12))
X = np.arange(0, 3, 0.1)

XYZ = np.random.uniform(size=(4000, k))
# print(XYZ)  # if you want to see it.
Z = XYZ.sum(axis=1)
sns.distplot(Z)
plt.plot(X, Zexact.pdf(X))
plt.show()
\end{minted}

\begin{minted}[]{R}
set.seed(3)

k = 3
X <- seq(0, 3, by = 0.1)
Zexact <- dnorm(X, mean = k / 2, sd = sqrt(k / 12))


XYZ <- matrix(NA, 4000, k)
for (i in 1:k) {
  XYZ[,i] <- runif(4000, min = 0, max = 1)
}
Z <- rowSums(XYZ)

par()
hist(Z, prob = TRUE, breaks = 31)
lines(X, Zexact, type = "l", col = "orange")
lines(density(Z), col = "blue")
\end{minted}


\begin{exercise}
What is the shape of \verb|XYZ| in the code above, i.e., how many rows and columns does it have? If you don't know, run the code, and print it.
\begin{solution}
\end{solution}
\end{exercise}

\begin{exercise}
What is the shape (rows and colums) of \verb|Z|?
\begin{solution}
\end{solution}
\end{exercise}

\begin{exercise}
Explain the values for \verb|loc|~ and \verb|shape| in \verb|Zexact|.
  (Read the documentation of scipy.stats.norm on the web is necessary.)
  To which definition in BH does this loc-scale transformation relate?
\begin{solution}
\end{solution}
\end{exercise}


\begin{exercise}
Change the seed to your student id, or any other number you like, run the code, and include the graph produced by your simulation.
Explain what you see.
\begin{solution}
\end{solution}
\end{exercise}


Now we do an exact computation.

\begin{minted}[]{python}
import numpy as np
from scipy.stats import norm

import matplotlib.pylab as plt
import seaborn as sns

sns.set()

N = 200
x = np.linspace(0, 2, 2 * N)
fx = np.ones(N) / N
f2 = np.convolve(fx, fx)
f3 = np.convolve(f2, fx)

k = 3

x = np.linspace(0, k, len(f3))
Zexact = norm(loc=k / 2, scale=np.sqrt(k / 12))


plt.plot(x, N * f3, label="conv")
plt.plot(x, Zexact.pdf(x))
plt.legend()
plt.show()
\end{minted}

\begin{minted}[]{R}
N = 200
x = seq(0, 2, length.out = 2 * N)
fx = rep(1, N) / N
f2 = convolve(fx, fx, type = "open")
f3 = convolve(f2, fx, type = "open")

k = 3

x = seq(0, k, length.out = length(f3))
Zexact = dnorm(x, mean = k/2, sd = sqrt(k / 12))

par()
plot(x, N * f3, col = "blue", type = "l", ylim = c(0, 0.8))
lines(x, Zexact, type = "l", col = "orange")
legend("topright", legend = "conv", bty = "n",
      lwd = 2, cex = 1.2, col = "blue", lty = 1)
\end{minted}

\begin{exercise}
Read the documentation of ~np.convolve~. Why is it called like this?
\begin{solution}
\end{solution}
\end{exercise}

\begin{exercise}
In the code, what is \texttt{f2}?
\begin{solution}
\end{solution}
\end{exercise}

\begin{exercise}
What is \texttt{f3}?
\begin{solution}
\end{solution}
\end{exercise}

\begin{exercise}
Why do we set \texttt{k=3}?
\begin{solution}
\end{solution}
\end{exercise}

\begin{exercise}
A bit harder, why do we plot \texttt{N*f3}, i.e., why do have to multiply with \texttt{N}? Relate this to the meaning of $f(x)\d x$, where $f$ the density of some random variable.
\begin{solution}
\end{solution}
\end{exercise}

\begin{exercise}
Yet a tiny bit harder, consider \texttt{f4 = np.convolve(f3, fx)} and \texttt{g4 = np.convolve(f2, f2)}. Why are they, numerically speaking,  equal?
\begin{solution}
\end{solution}
\end{exercise}

\begin{exercise}
When you would compute the maximum of \texttt{np.abs(f4 -g4)} you would see that this is about $10^{-10}$, or so.
Hence, a small number.
This is not equal to 0, but we know that this is due to rounding effects.

How can we use the function \texttt{np.isclose()} to get around this problem?
(You should memorize from this question that you should take care when testing on whether floating point numbers are the same or not.)
\begin{solution}
\end{solution}
\end{exercise}



\subsection{Challenges}
\label{sec:challenges-1}

\textbf{Week 3, Challenge 1. (Beluga)} \\
In this challenge, which is a continuation of~\cref{sec:coding-skills-1}, we discuss some ways to check whether $\V {N} \approx \V V \V v$ holds in general, and then we try to find a better approximation. We chopped up the challenge into many exercises, to help you organize the ideas.


Recall that in~\cref{sec:coding-skills-1} we have been a bit sloppy about the units, measuring the volumes of the airplane in $\m^{3}$ and a ping pong ball in $\cm^{3}$, so actually $N$ is in millions of ping pong balls.
Note that using different units can easily lead to  confusion; as a take-away , choose one unit.

One way to check the correctness of $\V N \approx \V V \V v$ is to change the scale. In fact, memorize that changing scale is an easy way to check laws.

\begin{exercise}
Suppose we instead measure the size of a ping pong ball in meters and the size of the airplane in hectometers.
Explain that $N$ is still in millions of ping pong balls.
What happens to $\V{N}$ and what happens to $\V V \V v$ (theoretically)?
\begin{solution}
$N$ does not change, so $\V N$ also does not change. On the other hand, $V$ and $v$ become 100 times as small, so their variances become $100^2$ times as small, so $\V V \V v$ becomes $10^8$ times as small.
\end{solution}
\end{exercise}


Another way to check a statement is to consider some extreme cases.

\begin{exercise} Suppose that we would know the size of a ping pong ball very accurately, i.e.  we consider the extreme case where $\V v \rightarrow 0$. Explain that the approximation is not a good approximation in this limit.
\begin{solution}
The approximation would predict that $\V N \to 0$ as well, but this is not correct, since there is also variability in $V$. In fact $\V N \approx \V V / (\E v)^2$, and the left hand side is not 0.
\end{solution}
\end{exercise}


\begin{exercise}
Which of these two checks convinces you most that something is wrong with this approximation, and why?
\begin{solution}
A theorem should hold for all cases, including the extreme cases. However, for an approximation it may be acceptable to fail in extreme cases, so this may not be the best check for an approximation. The first check shows better that we can be way off even in a real-life situation. \\
\end{solution}
\end{exercise}

We now turn to the task of trying to find a good approximation.

\begin{exercise} Assume that $X$ and $Y$ are independent. Show that
\begin{equation*}
\V {XY} = \V {X} \V {Y} + \V{X} \E {Y}^2 + \E {X}^2 \V{Y}.
\end{equation*}
\begin{solution}
By independence, we can split $ \E {(XY)^2}$ and $\E {XY}^2$. Hence, we find that
\begin{align*}
\V {XY} &= \E {(XY)^2} - \E {XY}^2 \\ &=  \E {X^2} \E {Y^2}  - \E {X}^2 \E {Y}^2
\\ &=  \E {X^2} \E {Y^2}  - \E {X}^2 \E {Y^2} + \E {X}^2 \E {Y^2} - \E {X}^2 \E {Y}^2
\\ &= \V {X} \E {Y^2} + \E {X}^2 \V{Y}
\\ &= \V {X} \V {Y} + \V{X} \E {Y}^2 + \E {X}^2 \V{Y}.
\end{align*}
\end{solution}
\end{exercise}

\begin{exercise} \label{ex:beluga5}
Assume in addition that we know at least one of $X$ and $Y$ quite precisely. Argue that the following is then a good approximation:
\begin{equation*}
\V {XY} \approx \V{X} \E {Y}^2 + \E {X}^2 \V{Y}.
\end{equation*}
\begin{solution}
If we know $X$ well, then $\V X \ll \E {X}^2$, so then  $\V X \V{Y} \ll  \E {X}^2 \V{Y}$. Similarly, if we know $Y$ well, then $\V Y \ll \E {Y}^2$, so then  $\V X \V{Y} \ll  \E {Y}^2 \V{X}$.
\end{solution}
\end{exercise}


So far we have only considered the variance of a product, but we would like to know the variance of a ratio.
%In \textbf{Mathematics II} you learned about Taylor expansions, which can be used to make accurate approximations.
For this we can use Taylor expansions to  make accurate approximations.

\begin{exercise}  \label{ex:beluga6}
Find the first order Taylor expansion of $\frac{1}{Z}$ around $a=  \E {Z}$. By taking the expectation and the variance of this expansion, show that
\begin{align*}
\E{\frac{1}{Z}} &\approx \frac{1}{\E{Z}}, & \V{\frac{1}{Z}} &\approx \frac{\V{Z}}{\E{Z}^4}.
\end{align*}
\begin{solution}
The first order Taylor expansion of $\frac{1}{Z}$ around $\E {Z}$ is given by $$\frac{1}{Z} = \frac{1}{\E {Z}} -  \frac{Z-\E Z}{\E {Z}^2} =  \frac{2}{\E {Z}} -  \frac{Z}{\E {Z}^2}.$$ Hence,
\begin{align*}
\E{\frac{1}{Z}} &\approx \E{\frac{2}{\E {Z}} -  \frac{Z}{\E {Z}^2}} = \frac{2}{\E{Z}}  -  \frac{\E{Z}}{\E {Z}^2}  = \frac{1}{\E{Z}} \\
\V{\frac{1}{Z}} &\approx \V{\frac{2}{\E {Z}} -  \frac{Z}{\E {Z}^2}} =  \V{-  \frac{Z}{\E {Z}^2}} = \frac{\V{Z}}{\E{Z}^4}.
\end{align*}
For the variance, note that adding a constant doesn't change the variance.
\end{solution}
\end{exercise}

\begin{exercise}
Combine all of the above to derive the following approximation for the variance of the ratio of two independent random variables $X$ and $Z$:
\begin{equation*}
\V {\frac{X}{Z}} \approx \frac{\V{X}}{\E{Z}^2} + \E {X}^2\frac{\V{Z}}{\E{Z}^4}.
\end{equation*}
\begin{solution}
Just plugging in $Y = \frac{1}{Z}$ to the result~\cref{ex:beluga5} and  using the results of \cref{ex:beluga6} yields the result.
\end{solution}
\end{exercise}


\begin{exercise}
Check this approximation in the ways of the first two exercises.
\begin{solution}
Multiplying both $X$ and $Z$ by a constant $c$ (e.g. 100) leaves the approximation invariant. In the limit $\V Z \rightarrow 0$, we obtain the exact answer (i.e. we get what we would if $Z$ would be a constant). In the limit $\V X \rightarrow 0$, we do not obtain the exact answer but also not something that is completely wrong (e.g. 0). \\
\end{solution}
\end{exercise}



% After doing all this work, we would of course like to know how well this approximation does. When comparing the approximation to the sample standard deviation found in~\cref{ex:2} for \texttt{num=500}, the result may be a bit disappointing. However, this is just because the sample standard deviation is also an estimate of the actual standard deviation of $N$, so by chance the result may be closer to $\V V \V v$ than to our new approximation.

%  In Chapter 10, you will learn something about the distribution of the sample variance. For now, just increase  \texttt{num}. We know this decreases the variance of the sample mean and it also decreases the variance of the sample variance, so we get a more accurate estimate.

% \begin{exercise} Use the result of the previous exercise to compute an approximation for $\V {N}  = \V {V/v}$. Also use the code with a (much) higher value of \texttt{num}, to show that the approximation $\V {N}  \approx \V V \V v$ is likely to be worse, even in the setting of~\cref{ex:2} where it was quite good.
% \begin{solution}
%   For the Beluga setting, the computed approximation of the standard deviation is 14.97 (more digits: 14.96688).
%   To say with some confidence that the approximation of exercise 7 is much better, one should take \texttt{num} to be at least 10000, although taking \texttt{num} to be at least 1000000 really shows that our hard work paid off.
%   \\
% \end{solution}
% \end{exercise}

% The following two exercises are really optional, but I found them very neat and insightful.



% \begin{exercise}
% Recall that for a nonnegative random variable $X$ with finite variance, we define the squared coefficient of variation as $ \mathrm{SCV}(X) =  \V {X} /\E {X}^2$. Using the SCV, show that the approximations of~\cref{ex:beluga5} and~\cref{ex:beluga6} can be rewritten in the following neat way:
% \begin{align*}
% \mathrm{SCV}(XY) &\approx \mathrm{SCV}(X) + \mathrm{SCV}(Y). \\
% \mathrm{SCV}\left(1/Z \right) &\approx \mathrm{SCV}(Z). \\
% \end{align*}
% \begin{solution}
% It is just algebra, but the result is nice.
% \end{solution}
% \end{exercise}


% In Chapter 10, you will learn Jensen's inequality, which implies that $\E{\frac{1}{Z}} \geq \frac{1}{\E{Z}}$ for all positive random variables $Z$. In the following exercise, we reflect on this by finding a more accurate approximation based on the second order Taylor expansion.

% \begin{exercise} Find the second order Taylor expansion of $\frac{1}{Z}$ around $a=  \E {Z}$.
% By taking the expectation, show that
% \begin{align*}
% \E{\frac{1}{Z}} \approx \frac{1}{\E{Z}} + \frac{2\V{Z}}{\E{Z}^3}.
% \end{align*}
% Note that this is always at least $\frac{1}{\E{Z}}$.
% \begin{solution}
% The second order Taylor expansion of $\frac{1}{Z}$ around $\E {Z}$ is given by \begin{align*} \frac{1}{Z} &= \frac{1}{\E {Z}} -  \frac{Z-\E Z}{\E {Z}^2} + \frac{2(Z-\E Z)^2}{\E {Z}^3} \\ &=  \frac{2}{\E {Z}} -  \frac{Z}{\E {Z}^2} + \frac{2(Z-\E Z)^2}{\E {Z}^3}.\end{align*} Hence, by linearity of expectation we conclude that
% \begin{align*}
% \E{\frac{1}{Z}} &\approx \E{\frac{2}{\E {Z}} -  \frac{Z}{\E {Z}^2} + \frac{2(Z-\E Z)^2}{\E {Z}^3}} \\ &= \frac{2}{\E{Z}}  -  \frac{\E{Z}}{\E {Z}^2} + \frac{2\E{(Z-\E Z)^2}}{\E {Z}^3}  = \frac{1}{\E{Z}} + \frac{2\V{Z}}{\E {Z}^3}.
% \end{align*}
% \end{solution}
% \end{exercise}


\end{document}
