\section{Lecture 3}

% \begin{enumerate}
% \item Why don't you setup a github repo to collaborate on making  solutions of the book?
% \item There is a \LaTeX\/ template in the sources directory on github to help you get started with the assignment.
% \end{enumerate}

% Three topics:
% \begin{enumerate}
% \item Prediction, and correlation
% \item Simulation
% \item Indicators
% \end{enumerate}
% \clearpage


\begin{exercise}
We ask a married woman on the street her height $X$.
What does this tell us about the height $Y$ of her spouse?
We suspect that taller/smaller people choose taller/smaller partners, so, given $X$, a simple estimator $\hat Y$ of $Y$ is given by
\begin{equation*}
  \hat Y = a X + b.
\end{equation*}
(What is the sign of $a$ if taller people tend to choose taller people as spouse?)
But how to determine $a$ and $b$? A common method is to find $a$ and $b$ such that the function
\begin{equation*}
  f(a,b) = \E{(Y-\hat Y)^2}
\end{equation*}
is minimized. Show that the optimal values are such that
\begin{align*}
  \hat Y = \E Y + \rho \frac{\sigma_Y}{\sigma_X} (X - \E X),
\end{align*}
where $\rho$ is the correlation between $X$ and $Y$ and where $\sigma_X$ and $\sigma_Y$ are the standard deviations of $X$ and $Y$ respectively.

\begin{solution}
We take the partial derivatives of $f$ with respect to $a$ and $b$, and solve for $a$ and $b$. In the derivation, we use that
\begin{align}
  \label{eq:6}
\rho = \frac{\cov{X,Y}}{\sqrt{\V X \V Y}} = \frac{\cov{X,Y}}{\sigma_X \sigma_Y} \implies  \rho \frac{\sigma_{Y}}{\sigma_{X}} = \frac{\cov{X,Y}}{\V X}.
\end{align}
Hence,
  \begin{align*}
f(a,b) &= \E{(Y-\hat Y)^2} \\
 &= \E{(Y-a X - b)^2} \\
 &= \E{Y^{2}} - 2a\E{YX} - 2b\E Y + a^{2}\E{X^2} + 2 ab \E X + b^{2}\\
\partial_{a} f &=-2 \E{YX} + 2a \E{X^2} + 2 b \E X = 0 \\
&\implies a \E{X^2} =  \E{YX}  -  b \E X \\
\partial_{b} f &=-2 \E{Y}  + 2 a \E X  + 2 b= 0 \\
& \implies  b = \E Y - a \E{X}\\
a \E{X^2} &=  \E{YX}  -  \E X (\E Y - a \E X) \\
&\implies  a (\E{X^{2}} - \E X \E X)  = \E{YX} - \E X \E Y  \\
&\implies a = \frac{\cov{X,Y}}{\V X} = \rho \frac{\sigma_Y}{\sigma_X}\\
b &= \E Y - \rho \frac{\sigma_Y}{\sigma_X}\E X\\
\hat Y &= a X + b \\
&= \rho \frac{\sigma_Y}{\sigma_X} X + \E Y - \rho\frac{\sigma_Y}{\sigma_X} \E X \\
&=  \E Y + \rho \frac{\sigma_Y}{\sigma_X} (X-\E X).
  \end{align*}
What a neat formula! Memorize the derivation, at least the structure. You'll come across many more optimization problems.

What if $\rho=0$?
\end{solution}
\end{exercise}

\begin{exercise}
Using scaling laws often can help to find errors. For instance,  the prediction $\hat Y$ should not change whether we measure the height in meters or centimeters.
In view of this, explain that
\begin{align*}
  \hat Y = \E Y + \rho \frac{\V Y}{\sigma_X} (X - \E X)
\end{align*}
must be wrong.
\begin{solution}
  If we measure $X$ in centimeters instead of meters, then $X$, $\E X$ and $\sigma_X$ are all multiplied by 100, and the prediction $\hat Y$ should also be expressed in centimeters But $\V Y $ scales as length squared.
  This messes up the units.
\end{solution}
\end{exercise}




\begin{exercise}
$N$ people throw their hat in a box. After shuffling, each of them takes out a hat at random. How many people do you expect to take out their own hat (i.e., the hat they put in the box); what is the variance? In BH.7.46 you have to solve this analytically. In the exercise here you have to write a simulator for compute the expectation and variance.
\begin{solution}
Let us first do one run.
\begin{pyblock}[][numbers=left,frame=lines]
import numpy as np

np.random.seed(3)

N = 4
X = np.arange(N)
np.random.shuffle(X)
print(X)
print(np.arange(N))
print((X == np.arange(N)))
print((X == np.arange(N)).sum())
\end{pyblock}
Here are the results of the print statements: $X = \py{X}$. The matches are \py{(X == np.arange(N))}; we see that $X[1] = 1$ (recall, python arrays start at index 0, not at 1, so $X[1]$ is the second element of $X$, not the first), so that the second person picks his own hat. The number of matches is therefore 1 for this simulation.

Now put the people to work, and let them pick hats for $50$ times.
\begin{pyblock}[][numbers=left,frame=lines]
import numpy as np

np.random.seed(3)

num_samples = 50
N = 5

res = np.zeros(num_samples)
for i in range(num_samples):
    X = np.arange(N)
    np.random.shuffle(X)
    res[i] = (X == np.arange(N)).sum()

print(res.mean(), res.var())
\end{pyblock}
Here is the number of matches for each round: \py{res}
The mean and variance are as follows: $\E X = \py{res.mean()}$ and $\V X = \py{res.var()}$.

For your convenience, here's the R code
\begin{minted}{R}
# set seed such that results can be recreated
set.seed(42)

# number simulations and people
numSamples <- 50
N <- 5

# initialize empty result vector
res <- c()

# for loop to simulate repeatedly
for (i in 1:numSamples) {

  # shuffle the N hats
  x <- sample(1:N)

  # number of people picking own hat (element by element the vectors x and
  # 1:N are compared, which yields a vector of TRUE and FALSE, TRUE = 1 and
  # FALSE = 0)
  correctPicks <- sum(x == 1:N)

  # append the result vector by the result of the current simulation
  res <- append(res, correctPicks)
}

# printing of observed mean and variance
print(mean(res))
print(var(res))
\end{minted}

\end{solution}
\end{exercise}
